\subsection{Introduction}

\begin{frame}{Bootloader role}
  \begin{itemize}
  \item The bootloader is a piece of code responsible for
    \begin{itemize}
    \item Basic hardware initialization
    \item Loading of an application binary, usually an operating
      system kernel, from flash storage, from the network, or from
      another type of non-volatile storage.
    \item Possibly decompression of the application binary
    \item Execution of the application
    \end{itemize}
  \item Besides these basic functions, most bootloaders provide a
    shell or menu
    \begin{itemize}
    \item Menu to select the operating system to load
    \item Shell with commands to load data from storage or network,
      inspect memory, perform hardware testing/diagnostics
    \end{itemize}
  \item The first piece of code running by the processor that can be
    modified by us developers.
  \end{itemize}
\end{frame}

\subsection{Booting on embedded platforms}

\begin{frame}{Booting on embedded platforms: ROM code}
  \begin{itemize}
  \item Most embedded processors include a {\bf ROM code} that
    implements the initial step of the boot process
  \item The ROM code is written by the processor vendor and directly
    built into the processor
    \begin{itemize}
    \item Cannot be changed or updated
    \item Its behavior is described in the processor datasheet
    \end{itemize}
  \item Responsible for finding a suitable bootloader, loading it and
    running it
    \begin{itemize}
    \item From NAND/NOR flash, from USB, from SD card, from eMMC, etc.
    \item Well defined location/format
    \end{itemize}
  \item {\em Generally} runs with the external RAM not initialized, so
    it can only load the bootloader into an internal SRAM
    \begin{itemize}
    \item Limited size of the bootloader, due to the size of the SRAM
    \item Forces the boot process to be split in two steps: first
      stage bootloader (small, runs from SRAM, initializes external DRAM),
      second stage bootloader (larger, runs from external DRAM)
    \end{itemize}
  \end{itemize}
\end{frame}

\begin{frame}{Booting on STM32MP1: datasheet}
  \begin{columns}
    \column{0.7\textwidth}
    \begin{center}
      \includegraphics[height=0.85\textheight]{slides/sysdev-bootloaders-sequence/stm32mp1-rom-code.png}
    \end{center}
    \column{0.3\textwidth}
    {\tiny
      Source: \url{https://www.st.com/resource/en/application_note/dm00389996-getting-started-with-stm32mp151-stm32mp153-and-stm32mp157-line-hardware-development-stmicroelectronics.pdf}\\
      Useful details: \url{https://wiki.st.com/stm32mpu/wiki/STM32_MPU_ROM_code_overview}
    }
  \end{columns}
\end{frame}

\begin{frame}{Booting on AM335x (32 bit BeagleBone): datasheet}
  \begin{columns}
      \column{0.6\textwidth}
      \begin{center}
        \includegraphics[height=0.85\textheight]{slides/sysdev-bootloaders-sequence/am335x-rom-code.png}
      \end{center}
      \column{0.4\textwidth}
      {\tiny
        Source:\\
        \url{https://www.mouser.com/pdfdocs/spruh73h.pdf},\\
	chapter 26
      }
    \end{columns}
\end{frame}

\begin{frame}{Two stage booting sequence}
  \begin{center}
    \includegraphics[height=0.85\textheight]{slides/sysdev-bootloaders-sequence/two-step-boot-process.pdf}
  \end{center}
\end{frame}

\begin{frame}{ROM code recovery mechanism}
  \begin{columns}
    \footnotesize
    \column{0.6\textwidth}
    \begin{itemize}
    \item Most ROM code also provide some sort of {\em recovery}
      mechanism, allowing to flash a board with no bootloader or a broken
      one, usually with a vendor-specific protocol over UART or USB.
    \item Often allows to push a new bootloader into RAM, making it
      possible to reflash the bootloader.
    \item Vendor-specific tools to run on the workstation
      \begin{itemize}
      \item STM32MP1: \href{https://www.st.com/en/development-tools/stm32cubeprog.html}{STM32 Cube Programmer}
      \item NXP i.MX: \href{https://github.com/NXPmicro/mfgtools}{uuu}
      \item Microchip AT91/SAM: \href{https://www.microchip.com/en-us/development-tool/SAM-BA-In-system-Programmer}{SAM-BA}
      \item Allwinner: \href{https://github.com/linux-sunxi/sunxi-tools}{sunxi-fel}
      \item Some open-source, some proprietary
      \end{itemize}
    \item Snagboot: new vendor agnostic tool replacing the above ones:
          \url{https://github.com/bootlin/snagboot}
    \end{itemize}
    \column{0.4\textwidth}
    \includegraphics[width=\textwidth]{slides/sysdev-bootloaders-sequence/stm32mp1-rom-code-recovery.pdf}
  \end{columns}
\end{frame}

\subsection{Bootloaders}

\begin{frame}{GRUB}
  \begin{columns}
    \column{0.6\textwidth}
    \begin{itemize}
    \item {\em Grand Unified Bootloader}, from the GNU project
    \item De-facto standard in most Linux distributions for x86
      platforms
    \item Supports x86 legacy and UEFI systems
    \item Can read many filesystem formats to load the kernel image,
      modules and configuration
    \item Provides a menu and powerful shell with various commands
    \item Can load kernel images over the network
    \item Also supports ARM, ARM64, RISC-V, PowerPC, but less popular
      than other bootloaders on those platforms
    \item \url{https://www.gnu.org/software/grub/}
    \item \url{https://en.wikipedia.org/wiki/GNU_GRUB}
    \end{itemize}
    \column{0.4\textwidth}
    \includegraphics[width=\textwidth]{slides/sysdev-bootloaders-sequence/grub2.png}
  \end{columns}
\end{frame}

\begin{frame}{U-Boot}
  \begin{columns}
    \column{0.7\textwidth}
    \begin{itemize}
    \item The de-facto standard and most widely used bootloader on
      embedded architectures: ARM, ARM64, RISC-V, PowerPC, MIPS, and
      more.
    \item Also supports x86 with UEFI firmware.
    \item Very likely the one provided by your SoC vendor, SoM vendor
      or board vendor for your hardware.
    \item We will study it in detail in the next section, and use it in
      all practical labs of this course.
    \item \url{https://www.denx.de/wiki/U-Boot}
    \end{itemize}
    \column{0.3\textwidth}
    \includegraphics[width=\textwidth]{slides/sysdev-bootloaders-sequence/u-boot.png}
  \end{columns}
\end{frame}

\begin{frame}{Barebox}
  \begin{columns}
    \column{0.7\textwidth}
    \begin{itemize}
    \item Another bootloader for most embedded CPU architectures:
      ARM/ARM64, MIPS, PowerPC, RISC-V, x86, etc.
    \item Initially developed as an alternative to U-Boot to address
      some U-Boot shortcomings
      \begin{itemize}
      \item {\em kconfig} for the configuration like the Linux kernel
      \item well-defined {\em device model} internally
      \item More Linux-style shell interface
      \item Cleaner code base
      \end{itemize}
    \item Actively maintained and developed, but
      \begin{itemize}
      \item Less widely used than U-Boot
      \item Less platform support than in U-Boot
      \end{itemize}
    \item \url{https://www.barebox.org/}
    \item Talk {\em barebox Bells and Whistles}, by Ahmad Fatoum, ELCE
      2020, \href{https://youtu.be/Oj7lKbFtyM0}{video} and
      \href{https://elinux.org/images/9/9d/Barebox-bells-n-whistles.pdf}{slides}
    \end{itemize}
    \column{0.3\textwidth}
    \includegraphics[width=\textwidth]{slides/sysdev-bootloaders-sequence/barebox.png}
  \end{columns}
\end{frame}